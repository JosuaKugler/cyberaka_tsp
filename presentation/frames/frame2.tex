\begin{frame}
    \frametitle{Grundlegende Definitionen}

\begin{columns}
\begin{column}{0.58\textwidth}

\only<1-3> {
    \begin{definition}[Kreis]
        Ist $(v_1, \dots, v_{n-1})$ mit $n > 1$ ein Pfad, so heißt $(v_0, \dots, v_{n-1}, v_0)$ Kreis.
    \end{definition}
}
\only<3>{
    \begin{definition}
        Einen Graphen ohne Kreise nennt man \textbf{azyklisch}.
    \end{definition}
}

\only<4->{
    \begin{definition}[Grad]
        %Sei $v\in V$ ein Knoten in einem Graphen $G = (V,E)$. Dann heißt $\# \{e \in E| v \in e\}$ der Grad von $v$.
        $d(v) = |\{e \in E| v \in e\}|$
    \end{definition}
    \begin{lemma}
        Die Anzahl der Knoten mit ungeradem Grad ist gerade.
    \end{lemma}
}

\end{column}
\begin{column}{0.43\textwidth}

    \only<1> {
    \begin{figure}
        \begin{tikzpicture}[scale=1.3, auto,swap]
        % First we draw the vertices
        \foreach \pos/\name in {{(0,2)/a}, {(0,0)/b}, {(1,-1)/c},
                                {(2,0)/d}, {(2,2)/e}}
        \node[vertex] (\name) at \pos {$\name$};
        % Connect vertices with edges and draw weights
        \foreach \source/ \dest /\weight in {b/a/1, c/b/2,
                                            c/d/3, d/e/5, e/a/6}
        \path[edge] (\source) -- node[font=\small] {} (\dest);
        \end{tikzpicture}
    \end{figure}
    }

    \only<2> {
    \begin{figure}
        \begin{tikzpicture}[scale=1.3, auto,swap]
        % Draw a 7,11 network
        % First we draw the vertices
        \foreach \pos/\name in {{(0,2)/a}, {(0,0)/b}, {(1,-1)/c},
                                {(2,0)/d}, {(2,2)/e}}
            \node[vertex] (\name) at \pos {$\name$};
        % Connect vertices with edges and draw weights
        \foreach \source/ \dest /\weight in {b/a/1, c/b/2,
                                            c/d/3, d/e/5, e/a/6}
            \path[edge] (\source) -- node[font=\small] {} (\dest);
            \begin{pgfonlayer}{background}
                \foreach \source / \dest in {b/a,c/b,d/c,d/e,e/a}
                \path[selected edge] (\source.center) -- (\dest.center);   
            \end{pgfonlayer}
        \end{tikzpicture}
    \end{figure}
    }

    

\only<3> {
\begin{figure}
		\begin{tikzpicture}[scale=1.3, auto,swap]
    % Draw a 7,11 network
    % First we draw the vertices
    \foreach \pos/\name in {{(0,2)/a}, {(0,0)/b}, {(1,-1)/c},
                            {(2,0)/d}, {(2,2)/e}}
        \node[vertex] (\name) at \pos {$\name$};
    % Connect vertices with edges and draw weights
    \foreach \source/ \dest /\weight in {b/a/1, c/b/2,
                                         c/d/3, d/e/5}
        \path[edge] (\source) -- node[font=\small] {} (\dest);
    
		\end{tikzpicture}
\end{figure}
}

\only<4>{
\begin{figure}
		\begin{tikzpicture}[scale=1.3, auto,swap]
    % Draw a 7,11 network
    % First we draw the vertices
    \foreach \pos/\name in {{(0,0)/a}, {(1,1)/b}, {(3,1)/c},
                            {(0,-2)/d}, {(2,0)/e}, {(1,-1)/f}, {(3,-1)/g}};
   	\node[vertex] (a) at (0,0) {$2$};
   	\node[vertex] (b) at (1,1) {$4$};
   	\node[vertex] (c) at (3,1) {$2$};
   	\node[vertex] (d) at (0,-2) {$4$};
   	\node[vertex] (e) at (2,0) {$4$};
   	\node[vertex] (f) at (1,-1) {$3$};
   	\node[vertex] (g) at (3,-1) {$3$};
   	
    % Connect vertices with edges and draw weights
    \foreach \source/ \dest /\weight in {b/a/1, c/b/2,d/a/4,d/b/2,
                                         e/b/3, e/c/5, d/g/2,
                                         f/d/4,f/e/1,
                                         g/e/3,g/f/5}
        \path[edge] (\source) -- node[font=\small] {} (\dest);
	\end{tikzpicture}
\end{figure}
}
\end{column}
\end{columns}
\end{frame}