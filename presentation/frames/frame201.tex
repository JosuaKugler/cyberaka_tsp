\begin{frame}
    \frametitle{Grundlegende Definitionen}
    Sei $S$ eine endliche Menge und $U \subseteq P(S)$ Familie von Teilmengen.
\onslide<2->{
    \begin{definition}
        Das Paar $M = \left( S, U \right) $ heißt \textbf{Matroid} und $U$ die Familie der \textbf{unabhängigen Mengen} von $M$, wenn gilt:
        \begin{enumerate}[1]
        		\item	$\emptyset \in U$
        		\item 	$A \in U, B \subseteq A  \implies B \in U$
        		\item 	$A, B \in U, \ \# B = \# A + 1 \implies \exists v\in B \setminus A \ \text{mit} \ A \cup \{ v\} \in U$
        \end{enumerate}
    \end{definition}
}
\onslide<3->{
    \begin{definition}
        Eine maximale unabhängige Menge heißt eine \textbf{Basis} des Matroids. Alle Basen enthalten die gleiche Anzahl von Elementen, der \textbf{Rang} $r(M)$ des Matroids.
    \end{definition}
}
\end{frame}